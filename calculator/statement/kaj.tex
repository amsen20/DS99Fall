\documentclass[11.5pt,a4paper,oneside]{article}
\usepackage{olympfa}
\renewcommand{\contestname}{%
فاینال -- تابستان بیست و نهمین دوره المپیاد کامپیوتر \\
۱ شهریور ۱۳۹۸\\
آزمون عملی سوم
}
\newcommand{\pass}{passwordforcontest}
\newcommand{\computer}{userforcontest}

\begin{document}
\def\problemCode{kaj}
\def\problemEnglishTitle{kaj}
\def\problemFarsiTitle{کاج}
\def\timeLimit{$3$ \second}
\def\memLimit{$256$ \megabytes}

\begin{problem}

در شهر آبابوا به دنباله، امیرمحمد می‌گویند، هم‌چنین به دنباله‌هایی که به ازای هر 
$i$
که 
1 < $a_i$
حداقل یکی از دو عدد
$a_i$
و 
$a_i-1$
سمت چپ مکان
$i$
ظاهر نشده باشند، ایمانی می‌گویند. (منظور از سمت چپ، تمام خانه‌های قبل از خانه $i$ است، نه تنها خانه‌ی قبلی)
\\
مثلا
 $<2, 2, 1>$ 
 یک امیرمحمد ایمانی معتبر است و
  $<1, 2, 2>$
   یک امیرمحمد ایمانی معتبر نیست.
\\
اهالی شهر آبابوا به سوالی مهم برخورده‌اند و از شما درخواست کمک دارند:
چند امیرمحمد ایمانی به طول
$n$
با اعداد
$1$
تا
$m$
داریم؟

\inputDescription

سطر اول وروی شامل دو عدد
$n$
و
$m$
است که
$n$
طول مورد‌نظر برای امیرمحمد ایمانی است
و امیرمحمد ایمانی فقط می‌تواند شامل اعداد
$1$
تا
$m$
باشد.
\\
\outputDescription

در خروجی باید تنها یک عدد چاپ کنید که جواب سوال مردم آبابوا است. از آنجا که این عدد می‌تواند خیلی بزرگ باشد باقی‌مانده آن بر $10^9 + 7$ را چاپ کنید.

\constraints
\begin{shortitems}
\item $1 \le n, m \le 500$
\end{shortitems}

\subtasks

\begin{subtasksTable}{|c|c|c|}
\hline
 \bf زیرمسئله   &   \bf نمره  &  \bf محدودیت‌ها  
\\ \hline
1 & $4$ & 
$n, m \le 8$ 
\\ \hline
2 & $32$ & 
$n, m \le 200$
\\ \hline
3 & $64$ &
بدون محدودیت اضافی
\\ \hline
\end{subtasksTable}


\sampleIO

\begin{example}
\exmp{%
2 2
}{%
4
}%
\exmp{%
3 2
}{%
6
}%
\end{example}

\sampleIODescription

در مثال اول، هر ۴ دنباله ممکن معتبر هستند.
\\
در مثال دوم، تنها ۲ دنباله
$<1, 2, 2>$
و
$<2, 1, 2>$
از ۸ حالت ممکن نامعتبر هستند، در نتیجه جواب برابر با ۶ است. 
\end{problem}

\end{document}
